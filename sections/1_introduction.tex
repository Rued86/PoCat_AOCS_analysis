\section*{Applicable Documents}

\begin{table}[H]
    \centering
    \resizebox{\columnwidth}{!}{
    \begin{tabular}{lll}
        \hline
        \textbf{Reference} & \textbf{Document Title} & \textbf{Issue/Release Date} \\
        \hline
        RD01 & ECSS-E-ST-31C Space Engineering & Issue 2, 2008/11/15 \\
        RD02 & ECSS-E-HB-31-01 - Space Engineering - Thermal design handbook - Part 1 to 16 & multiple \\
        RD03 & P. D.G. Gilmore, “Spacecraft Thermal Control Handbook: Fundamental Technologies”, ESA-TEC-MTT & 2002 \\
        \hline
    \end{tabular}
    }
\end{table}

\newpage

\section{Introduction}
The $^{\text{Po}}$Cat-Lektron Thermal Analysis is aimed to verify that all the components of the spacecraft
work in its operational temperature range during all mission phases. In order to develop
an accurate enough model, the following approach has been considered:

First, an isothermal solid cube will serve as an initial model. After that, models considering
radiation as the principal mode of heat transfer will provide more accurate results. Later
on, models considering both conduction and radiation as the main modes of heat transfer
will allow the needed characterization of the spacecraft to verify the thermal requirements
of each component.
\paragraph{}
In this document, the isothermal analysis and the radiation analysis of the $^{\text{Po}}$Cat
2 and $^{\text{Po}}$Cat3 satellites are presented.

The main objective of the isothermal analyses is to identify a passive thermal control
which maintains the spacecraft temperature in the correct behavior. In this case
, the evaluated technique is the use of different paints on the spacecraft surface.
\paragraph{}
As for the radiation analyses, they are aimed to validate the spacecraft thermal model
and verify its behavior under different heating environments.

\subsection{The \texorpdfstring{\textsuperscript{Po}}CCat-Lektron mission}

\subsubsection{Mission statement}

The $^{\text{Po}}$Cat-Lektron is a mission resulting from the IEEE OpenPocketQube Kit initiative, 
developed at the UPC NanoSat Lab. The mission has been selected in the 4th call of 
the ESA Fly Your Satellite! (FYS) program. The mission analysis presented corresponds 
to the $^{\text{Po}}$Cat-Lektron mission. It consist of two 1P PocketQubes, the PoCat-2 and the 
PoCat-3, developed as a part of the IEEE OpenPocketQube Kit. This mission aims to 
demonstrate the feasibility of PocketQube platforms for remote sensing applications.\cite{wiki}
\paragraph{}
The payloads on board of the PocketQubes are two passive radiometers to be use for 
RFI purposes on K and L bands. Apart from the remote sensing nature of the mission, 
this mission also aims to demonstrate the feasibility of the PocketQube platforms to 
create, manage and join Federated Satellite Systems. To do so, the FSS Experiment 
will be reproduced as a part of the experiments of the mission.

\subsubsection{Mission Objectives}

\begin{itemize}
    \item \textbf{Demonstration of Scientific Viability:} Demonstrate the feasibility of 
    conducting scientific missions using PocketQube platforms. To do so, the mission
    proposes collecting valuable RFI data through a K-Band and L-Band passive 
    radiometers (One for each PocketQube). The payloads will monitor interferences on 
    these bands. This data will facilitate enhanced detection and the generation of
    heatmaps indicating RFI distribution across the globe. In this experiment we aim
    to obtain data on the K-Band to see the impact on the atmospheric water vapor 
    measurements, and in the L-Band the interferences over the Position Navigation 
    and Timing (PNT) signals.

    \item \textbf{Satellite Federation Concept:} To establish and demonstrate that 
    PocketQube platforms can create, manage and join Federated Satellite System 
    (FSS). This proof of concept for this resource-limited platforms is based on 
    the reproduction of the FSS Experiment conducted at the UPC NanoSat Lab. The 
    demonstration consist on create a federation between 2 PocketQubes, in order 
    to download data. Previous missions such as the FSS-Cat from the UPC NanoSat 
    Lab demonstrated the feasibility of this opportunistic collaboration using 6U 
    CubSats.

    \item \textbf{Educational Development:}  As a mission developed at the UPC NanoSat
     Lab, the mission is oriented for undergraduated students to gain experience and 
     get involved in real space missions. In addition, several Bachelor and Master 
     Thesis had been done from this project, apart from the academic papers that 
     this project has produced.

\end{itemize}

