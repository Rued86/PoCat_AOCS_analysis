\section{Worst Hot and Cold Case Definition}

\subsection{External Heating Environment}
\textsuperscript{Po}Cat-2 and \textsuperscript{Po}Cat-3 are expected
to be launched together, from a yet to define launcher. They will perform,
ideally, a polar or quasipolar orbit with near-zero eccentricity. 
In table \ref{tab::orbit_params} the most critical external heating 
environment orbit parameters taken in the analyses present in this document
 (unless otherwise specified).

 \paragraph{}
 Do note that as of the latest revision of this document the selected launcher and
 final orbit parameters are still an unknown matter. Therefore, within reasonable bounds,
 the calculations will be presented considering the extremal ranges of these boundary
 conditions. 

 \begin{table}[H]
    \centering
    \begin{tabular}{lc}
        \toprule
        \textbf{Parameter} & \textbf{Value} \\
        \midrule
        Perigee height             & 450--500 km     \\
        Eccentricity ($e$)         & 0       \\
        Inclination ($i$)          & 80--100       \\
        Estimated launch date      & 2026--Q2        \\
        \bottomrule
    \end{tabular}
    \caption{External heating environment orbit parameters.}
    \label{tab::orbit_params}
\end{table}

Extreme hot and cold heating environments have been defined in order to envelop all possible cases in between.
Table \ref{tab:heating_environments} summarizes the design hot and cold environments where extreme and less pessimistic values suggested 
in RD03 (2-$\sigma$) have been taken respectively. In this table, the \textbf{solar flux} maximum 
and minimum taken values have been those corresponding to the winter and summer solstices respectively as specified in RD03.

\paragraph{}

Regarding the \textbf{Earth albedo} and the \textbf{Earth Infrared}, values from RD03 have been taken, using a 2-$\sigma$ margin. The time
period taken is of 90 minutes, which approximately coincides with orbit time, neglecting variations on lower time scales. The appropiate 
non-Lambertian corrections have been applied as specified in the reference document.

\paragraph{}

\textbf{The duration of the eclipses} can also have a high impact on the maximum and minimum temperatures reached by 
the satellite during an orbit. The most extreme cases (i.e., the shortest and longest eclipse times) have been 
considered for the orbit inclination shown in Table~3. To this aim, the so-called \textbf{orbit beta angle} $\beta$ 
has been defined for each case as the minimum angle between the orbit plane and the solar vector. For the cold case,
the beta angle is 0$^\circ$ since the longest eclipse occurs when the orbit plane is parallel to the Ecliptic. As for
the hot case, the shortest eclipse occurs when the value of $\beta$ is closest to $\pm$90$^\circ$. Since the Earth 
axial tilt ($\epsilon_E$) is 23.44$^\circ$, the maximum beta angle can be obtained as the sum of the Earth tilt angle 
and the spacecraft orbit inclination as:

\begin{equation}
    \beta_{\text{max}} = \varepsilon_E + i
\end{equation}

Despite this, due to the inclination being yet unknown, the most extreme values are taken.

\begin{table}[H]
    \centering
    \begin{tabular}{lcccc}
        \toprule
        \textbf{Heating Environment} & \textbf{Solar Irradiance (W/m$^2$)} & \textbf{Earth Albedo} & \textbf{Earth IR (W/m$^2$)} & \textbf{$\beta$ ($^\circ$)} \\
        \midrule
        Hot Case             & 1414 & 0.55  & 219 & 90 \\
        Cold Case            & 1322 & 0.22  & 238 & 0     \\
        Extreme Hot Case   & 1414 & 0.8 & 261 & 90 \\
        Extreme Cold Case  & 1322 & 0.05 & 189 & 0 \\
        \bottomrule
    \end{tabular}
    \caption{Design hot and cold external heating environments}
    \label{tab:heating_environments}
\end{table}

It needs to be noted that the later values presented in Table \ref{tab:heating_environments} represent extreme (eventually non-physically possible) situations aiming to envelope the hottest and the coldest possible heating environments the spacecraft will encounter. The non-extreme parameters of the table will be the ones used to perform the thermal analysis of the spacecraft.


\subsection{Satellite Internal Heat Dissipation and Attitude}


The satellite operating modes defined, from the internal heat dissipation point of view, are the following:

\begin{itemize}
    \item \textbf{Standby (Sb)}: Period before the satellite is turned on. All subsystems are inactive.
    \item \textbf{Released (R)}: After the standby period, the satellite is turned on, with the EPS and OBC as the only active subsystems.
    \item \textbf{Pre-detumbling (PD)}: Once the COMMS antenna haas been deployed, the EPS, OBC, COMMS and AOCS (only determination for telemetry) are active. The COMMS subsystem is transmitting data with a ratio of 1\% of the orbit time.
    \item \textbf{Detumbling (D)}: Same as pre-detumbling state, with the magnetorquers operating.
    \item \textbf{Detumbled (Dd)}: Once the satellite is detumbled, the AOCS is keeping the desired attitude. The COMMS subsystem is transmitting data with a ratio of 1\% of the orbit time.
    \item \textbf{Nominal (N)}: Satellite is fully operative. The payload is executed a yet to define number of times per orbit. The COMMS subsystem is transmitting data with a ratio of 1\% of the orbit time.
    \item \textbf{Contingency (C)}: When the batteries fall below a certain value. In this mode, some functionalities of the flight software are disabled (P/L, Nadir, FSS).
    \item \textbf{Survival (S)}: After an unexpected anomaly in battery levels, the satellite enters this mode. Transmissions are ceased.
    \item \textbf{Satellite Off (OFF)}: Satellite turned off during its operation due to unexpected events. All subsystems are inactive.
\end{itemize}

As for the satellite attitude, the following states have been considered:

\begin{itemize}
    \item \textbf{Nadir Pointing (NP)}: Satellite -Z face pointing towards Earth center. Desired (ideal) attitude of the spacecraft achieved by the AOCS during its operation.
    \item \textbf{Random Rotation (RR)}: Satellite randomly rotating about its three axes (e.g., during the deployment).
    \item \textbf{Zenith Pointing (ZP)}: Satellite +Z face pointing towards Earth center (-Z face pointing towards zenith). Critical situation considered, as a worst-case assumption.
\end{itemize}

The mean power consumption of each satellite operating mode, and the satellite attitude state in which they can occur, are summarized in Table \ref{tab:satellite_modes}.
Mean power consumption values have been extracted from the power budget yet to be presented.


\begin{table}[H]
    \centering
    \begin{tabular}{cccc}
        \toprule
        \textbf{Operating Mode} & \textbf{Power Consumption (mW)} & \textbf{Attitude State} & \textbf{COMMS Config.} \\
        \midrule
        \multicolumn{4}{c}{\textbf{Commissioning Phase}} \\
        Standby (Sb)        & 0 & RR             & Stowed   \\
        Released (R)        & 74 & RR             & Stowed   \\
        Pre-detumbling (PD) & 497 & RR             & Deployed   \\
        Detumbling (D)      & 995 & RR             & Deployed   \\
        Detumbled (Dd)      & 497 & NP     & Deployed   \\
        \midrule
        \multicolumn{4}{c}{\textbf{Operational Phase}} \\
        Nominal (N)         & 497 & NP        & Deployed \\
        Contingency (C)     & 77 & NP      & Deployed \\
        Survival (S)        & 46 & NP       & Deployed \\
        Satellite Off (OFF) & 0 & NP       & Deployed \\
        \bottomrule
    \end{tabular}
    \caption{Satellite operating modes, mean power consumption and attitude states.}
    \label{tab:satellite_modes}
\end{table}
Do note that, after the release, the spacecraft will enter an initial (Init) state (Released) where it'll begin the deployment
process of the COMMS antenna, only entering the Nominal mode after communication has been established with the 
Ground Station.

\paragraph{}

In this thermal analysis only worst hot and cold cases have been studied, enveloping all possible cases in between.
From a thermal point of view, the RR attitude state is less critical than the NP and ZP states; since the satellite
is constantly rotating, the heat flux is expected to be averaged between the different faces.
Finally, it is not clear whether the deployed or the stowed configuration corresponds to the worst hot and cold case. 
Despite this, due to the limited size of the COMMS Antenna, it is expected to find a non significative difference, and,
therefore, this analysis will only evaluate the deployed case.
\paragraph{}
From the satellite modes during the operational phase, only Nominal and Survival modes will be analyzed, since these are the modes 
with the higher and lower power consumption respectively. It is expected that the Survival mode will be of most criticality due to
the shutting off of the battery heater, in favour of a lower power consumption.

\paragraph{}
From the attitude point of view, the NP state will be analyzed. The ZP state might be considered in future analysis if requiered. This is
due to the fact that this mode is not expected to present a relevant probability of occurance.

\subsection{Worst Hot and Cold Cases}

Having defined the hot and cold heating environments and the hot and cold satellite internal heat dissipation and attitude
cases, the overall worst cases are summarized in Table \ref{tab:worst_cases}.

\begin{table}[H]
    \centering
    \begin{tabular}{ccccc}
        \toprule
        & \textbf{Operating Mode} & \textbf{Attitude State} & \textbf{COMMS Config.} & \textbf{Heating Environment} \\
        \midrule
        \multicolumn{5}{c}{\textbf{Prior to Detumbling}} \\
        Worst Hot Case  & D & RR & Deployed & Hot Case  \\
        Worst Cold Case & Sb  & RR & Deployed & Cold Case \\
        \midrule
        \multicolumn{5}{c}{\textbf{Operational}} \\
        Worst Hot Case  & N  & NP & Deployed & Hot Case  \\
        Worst Cold Case & SS & NP & Deployed & Cold Case \\
        \bottomrule
    \end{tabular}
    \caption{Worst Hot and Cold Cases.}
    \label{tab:worst_cases}
\end{table}